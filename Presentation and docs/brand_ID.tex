%% LyX 2.2.3 created this file.  For more info, see http://www.lyx.org/.
%% Do not edit unless you really know what you are doing.
\documentclass[english]{article}
\usepackage[T1]{fontenc}
\usepackage{longtable}
\usepackage{float}
\usepackage{graphicx}

\makeatletter

%%%%%%%%%%%%%%%%%%%%%%%%%%%%%% LyX specific LaTeX commands.
%% Because html converters don't know tabularnewline
\providecommand{\tabularnewline}{\\}

\makeatother

\usepackage{babel}
\begin{document}

\section{Brand Identity and Design}

An important aspect of the project development concerned the design
of a Brand Identity that characterized in an unique fashion the final
product delivered. The inspirational colors and shapes of the whole
front-end and presentation stack resembles the undeniable suggestion
of confidence that IBM first-rate services offer. The choice to include
a Brand Identity chapter in this document stems from the fact that,
due to the high complexity of the evaluation algorithm, a good visualization
mask needed to be developed in order to convey the robustness and
sophistication that the algorithmic part present, while enabling not
experts of the Machine Learning field to understand the potential
of this tool.

Certainly, a future possible development is the production (thus public
distribution) of the platform as example of combined AI technologies
applied to the real life.

\subsection{Colors shades}

\begin{figure}[H]
\begin{centering}
\includegraphics[scale=0.3]{\string"/Users/phil/git/FakeNews/Presentation and docs/colors\string".png}
\par\end{centering}
\caption{Palette of colors used on the front-end/presentation layer of the
project}
\end{figure}

The first choice that the UX unit has taken was to generate a palette
of colors on which would be later drawn the template and the presentation
layout. 

As described before, the original colors were taken from IBM Corporate
Identity and readapted for the project purposes
\begin{center}
\begin{longtable}{|c|c|c|c|}
\hline 
\textbf{Color} & \textbf{Name} & \textbf{HTML HEX} & \textbf{RGB}\tabularnewline
\hline 
\includegraphics[bb=0bp 0bp 100bp 100bp,scale=0.07]{\string"/Users/phil/git/FakeNews/Presentation and docs/colors/022231\string".png} & Daintree & \#022231 & 2,34,49\tabularnewline
\hline 
\includegraphics[bb=0bp 0bp 100bp 100bp,scale=0.07]{colors/022b48} & Green Vogue & \#022b48 & 2,43,72\tabularnewline
\hline 
\includegraphics[bb=0bp 0bp 100bp 100bp,scale=0.07]{colors/0c3d5f} & Tarawera & \#0c3d5f & 12,61,96\tabularnewline
\hline 
\includegraphics[bb=0bp 0bp 100bp 100bp,scale=0.07]{colors/0e486f} & Chathams Blue & \#0e486f & 14,72,111\tabularnewline
\hline 
\includegraphics[bb=0bp 0bp 100bp 100bp,scale=0.07]{colors/0d6797} & Blue Chill & \#0d6797 & 13,103,151\tabularnewline
\hline 
\end{longtable}
\par\end{center}

\subsection{Brand creation}

Inspired by the name of IBM's utilized service ``Watson'', we appeal
to our project the name ``Dr.Watson'', a fictional character in
the Sherlock Holmes stories by Sir Arthur Conan Doyle, while \emph{GestIT}
remains the group formal name. 

The brand creation begun with logo's designing. Such process included
a through research of the main font, later found as \emph{Merriweather
Black} and the application of the Palette colors.

\begin{figure}[H]
\begin{centering}
\includegraphics[scale=0.15]{\string"/Users/phil/git/FakeNews/Presentation and docs/logo_grid\string".png} 
\par\end{centering}
\caption{First sketch of ``W'' which stands for Watson of ``Dr.Watson''}
\end{figure}

\begin{figure}[H]
\begin{centering}
\includegraphics[scale=0.075]{\string"/Users/phil/git/FakeNews/Presentation and docs/logo_shadows\string".png} 
\par\end{centering}
\caption{Final ``W'' with transparent background}
\end{figure}

\begin{figure}[H]
\begin{centering}
\includegraphics[scale=0.07]{\string"/Users/phil/git/FakeNews/Presentation and docs/logo_shadows_bg\string".png} 
\par\end{centering}
\caption{Final ``W'' with dark blue background}
\end{figure}

\end{document}
